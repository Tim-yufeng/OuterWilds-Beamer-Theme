% Compile with XeLaTeX
\documentclass[aspectratio=169]{beamer}

% 加载我们刚刚完成的“挪麦档案”主题
\usetheme{OuterWilds}

% --- 建议配置:学术字体 (可选) ---
% 如果您电脑上有这些字体,取消注释会更有味道
% \usepackage{fontspec}
% \setmainfont{Roboto} % 或 Arial
% \setmonofont{Fira Code} % 或 Courier New

% --- 1. 档案元数据 (Metadata) ---
\title{Chrono-Spatial Analysis of the Outer Wilds System}
\subtitle{A Case Study on Macroscopic Quantum Phenomena and Time Loop Topology}
\author{Hearthian Research Division}
\institute{Outer Wilds Ventures \\ Timber Hearth Observatory}
\date{Cycle 9,354}

\begin{document}

% --- 2. 封面页 (悬浮取景框风格) ---
% 使用 plain 选项去除页眉页脚
\begin{frame}[plain]
    \titlepage
\end{frame}

% --- 3. 目录页 ---
\begin{frame}{Expedition Log}
    \tableofcontents
\end{frame}

% =========================================================
% SECTION 1: 触发酷炫的过渡页 (宇宙之眼背景 + 巨大数字)
% =========================================================
\section{Introduction to the System}

% --- 4. 列表展示页 (测试几何符号) ---
\begin{frame}{Research Objective: The Solar System}
    We aim to map the celestial bodies and anomalies of the local star system before the impending stellar event.

    \vspace{0.5em}
    
    \begin{itemize}
        \item \textbf{Primary Targets}
        \begin{itemize}
            \item \textbf{The Hourglass Twins}: Sand transfer dynamics and high-energy readings.
            \item \textbf{Brittle Hollow}: Crustal stability and the black hole singularity.
            \item \textbf{Dark Bramble}: Spatial distortion and biological hazards.
        \end{itemize}
        \item \textbf{Key Anomalies Detected}
        \begin{itemize}
            \item Rapid stellar evolution (Supernova imminent).
            \item Temporal discontinuities (Time Loop phenomenon).
        \end{itemize}
    \end{itemize}
\end{frame}

% =========================================================
% SECTION 2: 再次触发过渡页 (进度条动态增长)
% =========================================================
\section{Temporal Dynamics}

% --- 5. 区块展示页 (测试极简 Block) ---
\begin{frame}{The Ash Twin Project (ATP)}
    The ATP acts as the central processing unit for the memory transmission protocol.

    % 普通区块:极淡蓝色背景 + 左侧粗线
    \begin{block}{Hypothesis: Memory Transmission}
        Information is sent back in time by exactly 22 minutes using the energy of a supernova, effectively violating causality to preserve knowledge.
    \end{block}

    \vspace{1em}

    % 警示区块:橙色线条
    \begin{alertblock}{Critical Warning: Supernova Event}
        The sun has reached the end of its main sequence life cycle. Total system annihilation is guaranteed at $T=22:00$.
    \end{alertblock}
\end{frame}

% --- 6. 图文排版页 ---
\begin{frame}{Topological Structure of the Loop}
    \begin{columns}
        \column{0.6\textwidth}
        The time loop can be modeled as a closed timelike curve (CTC). The \textbf{Advanced Warp Core} inside Ash Twin is responsible for opening the singularity.
        
        \[
            \oint_{C} ds^2 < 0 \quad \text{where } \Delta t = -22 \text{ min}
        \]
        
        \begin{itemize}
            \item \textbf{Trigger}: Sun Station (Failed).
            \item \textbf{Power Source}: Natural stellar death (Success).
        \end{itemize}

        \column{0.4\textwidth}
        % 这里可以用 exampleblock 来模拟计算机终端显示
        \begin{exampleblock}{Status Report}
            \ttfamily
            > WARP CORE: ACTIVE \\
            > MASKS: PAIRED \\
            > MEMORY: UPLOADING...
        \end{exampleblock}
    \end{columns}
\end{frame}

% =========================================================
% SECTION 3: 最后一次过渡
% =========================================================
\section{Quantum Mechanics}

% --- 7. 量子规则页 ---
\begin{frame}{Macroscopic Quantum Rules}
    Unlike standard physics, quantum phenomena in this system are observable on a planetary scale (e.g., the Quantum Moon).

    \begin{block}{The Rule of Quantum Imaging}
        "Observing an image of a quantum object is equivalent to observing the object itself."
    \end{block}

    \begin{itemize}
        \item \textbf{Entanglement}: Standing on a quantum object allows the observer to become quantum.
        \item \textbf{The Sixth Location}: The Quantum Moon exists in a superposition of states around all 5 planets and the Eye.
    \end{itemize}
\end{frame}

% --- 8. 结论页 ---
\begin{frame}{Conclusion: The Eye of the Universe}
    The Eye is older than the universe itself. It acts as a macroscopic observer, potentially collapsing all possibilities into a new reality.
    
    \vspace{2em}
    \centering
    \large \textbf{Final Consensus} \\
    \vspace{0.5em}
    \normalsize
    The search for the Eye is not about saving this universe, but birthing the next.
\end{frame}

% --- 9. 致谢页 (信号中断风格) ---
% 直接调用命令,不要包裹在 frame 中!
\thankframe

\end{document}